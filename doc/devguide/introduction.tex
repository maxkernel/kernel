\section{Introduction}
Hello and welcome to MaxKernel. As a developer, I know you have lots of questions. How do I compile and install? How do I configure and calibrate? How do I write a new module? Let's explore these questions together in a bit, but first lets have a heart to heart chat. The intent of this document is to answer as many of these questions as possible. With any comprehensive software project, this is very hard indeed. Since we are also a 100\% opensource operation, that makes it specifically difficult as we developers don't like to be bothered with silly documentation. If you see an error, misprint, or have a general comment please email the authors, they'd love to hear from you. Or better yet, you can contribute by submitting the fix yourself. Opensource means 100\% user (non-propreitary) contribution and we rely on your support not only in software code, but in docuemtation as well. Thank you (but only if you have contributed).

\subsection{License}
MaxKernel is licensed under the GNU GPL.

\subsection{Donations}
MaxKernel has some financial overhead (server costs, domain name renewal, robotic testing equipment, etc.). Any donations would be greatly appreciated and would go towards making the software more accessable. Please visit the website for details on how to contribute. Thank you (but only if you have donated).

\subsection{Origin}
MaxKernel came about after three years of robotic research and tinkering. Just over the last half-decade, robotics research has exploded in academia, industry, and among do-it-yourself enthusiasts. Amid the shrapnel, supporting libraries have begun to emerge that act as a foundational layer and unify seperate modules into a common "information pipeline". Seperate modules, such as a jpeg compressor, can be "stitched" into a coherent application as long as the module conforms to a certain interface. The breadboard paradigm, where each physical chip on the breadboard is a software module and the links between the chips are created by the foundational layer works well when talking about this type of robotic library.



 With increasingly compact sensors and hardware, we are approaching a critical mass of new technologies that will enable us to drive low-cost and diverse industrial and commercial markets. As evident by growing robotic market share in autonomous vacuum cleaners, consumers are hungry for the latest robotic technologies. However, robotic research is still in it's infancy compared to other technological mature markets. In contrast, standalone applications for computers are very successful using vertical integration strategies. For these, overall objectives rarely change and therefore, developers are free to tightly integrate disparate components and turn out a complete top-to-bottom solution.

In the field of robotics research, one cannot set objectives so tightly. In addition, sensor technologies are constantly changing along with their application interfaces. It is very difficult to build reusable modules. And if developing a system to accommodate constantly changing hardware and ever-changing objectives weren't hard enough, new algorithms for dealing with sensors and control systems (eg. Kalman filters, rapidly-exploring random trees, etc.) are evolving just as rapidly.

To satisfy this demand, a simple bare-minimum framework was designed to accomplish each of these goals. MaxKernel was born.


\subsection{Competitors}


